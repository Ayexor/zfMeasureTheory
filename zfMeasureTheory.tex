% !TeX spellcheck = en_GB
\input{./header.tex}

\author{Roy Seitz}
\title{Analysis 3 -- Measure Theory\\Summary}
\begin{document}
\maketitle
\tableofcontents

\section{Measure Spaces}
\subsection{Algebras and $\sigma$-Algebras of Sets}

\begin{definition}
  Limes superior and inferior for sets:
  \[
  \limsup_{n \to \infty} A_n = \bigcap_{n=1}^\infty \bigcup_{m=n}^\infty A_m
  \qquad\text{and}\qquad
  \liminf_{n \to \infty} A_n = \bigcup_{n=1}^\infty \bigcap_{m=n}^\infty A_m.
  \]
\end{definition}

\begin{definition}
  Let $\mathcal A \subset \mathcal P(X)$.
  $\mathcal A$ is called an \textbf{Algebra in $X$} if
  \[
    x \in \mathcal A,
    \qquad
    \forall A, B \in \mathcal A\colon A \cup B \in \mathcal A
    \qquad\text{and}\qquad
    \forall A \in \mathcal A\colon A^c \in \mathcal A.
  \]
  An Algebra is called a \textbf{$\sigma$-Algebra} if for any sequence $(A_n)_n \in \mathcal A$ we have
  $\bigcup_{n=1}^\infty A_n \in \mathcal A$.
\end{definition}

\begin{definition}
  Let $K \subset \mathcal P(X)$.
  The intersection of all $\sigma$-algebras including $K$ is called the
  \textbf{$\sigma$-algebra generated by $K$} and will be denoted by $\sigma(K)$.
\end{definition}

\begin{definition}
  Let $(X, d)$ be a metric space.
  The $\sigma$-algebra generated by the open sets is called the
  \textbf{Borel $\sigma$-algebra} of $X$, denoted by $\mathcal B(X)$.
  The elements of $\mathcal B(X)$ are called \textbf{Borel sets}.
\end{definition}

%\section{Test}
\subsection{Measures}

\begin{definition}
  Let $\mathcal A \subset \mathcal P(X)$ be an algebra
  and $\mu\colon \mathcal A \to [0, \infty]$ with $\mu(\emptyset) = 0$.
  We call $\mu$ \textbf{additive} if for any mutually disjoint family $A_1, \ldots, A_n \in \mathcal A$
  \[\mu \left(\bigcup_{k=1}^n A_k \right) = \sum_{k=1}^n \mu(A_k).\]
  We call $\mu$ \textbf{$\sigma$-additive} if that equation also holds for countable families of sets.
\end{definition}

\end{document}

